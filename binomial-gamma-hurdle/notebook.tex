
% Default to the notebook output style

    


% Inherit from the specified cell style.




    
\documentclass[11pt]{article}

    
    
    \usepackage[T1]{fontenc}
    % Nicer default font (+ math font) than Computer Modern for most use cases
    \usepackage{mathpazo}

    % Basic figure setup, for now with no caption control since it's done
    % automatically by Pandoc (which extracts ![](path) syntax from Markdown).
    \usepackage{graphicx}
    % We will generate all images so they have a width \maxwidth. This means
    % that they will get their normal width if they fit onto the page, but
    % are scaled down if they would overflow the margins.
    \makeatletter
    \def\maxwidth{\ifdim\Gin@nat@width>\linewidth\linewidth
    \else\Gin@nat@width\fi}
    \makeatother
    \let\Oldincludegraphics\includegraphics
    % Set max figure width to be 80% of text width, for now hardcoded.
    \renewcommand{\includegraphics}[1]{\Oldincludegraphics[width=.8\maxwidth]{#1}}
    % Ensure that by default, figures have no caption (until we provide a
    % proper Figure object with a Caption API and a way to capture that
    % in the conversion process - todo).
    \usepackage{caption}
    \DeclareCaptionLabelFormat{nolabel}{}
    \captionsetup{labelformat=nolabel}

    \usepackage{adjustbox} % Used to constrain images to a maximum size 
    \usepackage{xcolor} % Allow colors to be defined
    \usepackage{enumerate} % Needed for markdown enumerations to work
    \usepackage{geometry} % Used to adjust the document margins
    \usepackage{amsmath} % Equations
    \usepackage{amssymb} % Equations
    \usepackage{textcomp} % defines textquotesingle
    % Hack from http://tex.stackexchange.com/a/47451/13684:
    \AtBeginDocument{%
        \def\PYZsq{\textquotesingle}% Upright quotes in Pygmentized code
    }
    \usepackage{upquote} % Upright quotes for verbatim code
    \usepackage{eurosym} % defines \euro
    \usepackage[mathletters]{ucs} % Extended unicode (utf-8) support
    \usepackage[utf8x]{inputenc} % Allow utf-8 characters in the tex document
    \usepackage{fancyvrb} % verbatim replacement that allows latex
    \usepackage{grffile} % extends the file name processing of package graphics 
                         % to support a larger range 
    % The hyperref package gives us a pdf with properly built
    % internal navigation ('pdf bookmarks' for the table of contents,
    % internal cross-reference links, web links for URLs, etc.)
    \usepackage{hyperref}
    \usepackage{longtable} % longtable support required by pandoc >1.10
    \usepackage{booktabs}  % table support for pandoc > 1.12.2
    \usepackage[inline]{enumitem} % IRkernel/repr support (it uses the enumerate* environment)
    \usepackage[normalem]{ulem} % ulem is needed to support strikethroughs (\sout)
                                % normalem makes italics be italics, not underlines
    

    
    
    % Colors for the hyperref package
    \definecolor{urlcolor}{rgb}{0,.145,.698}
    \definecolor{linkcolor}{rgb}{.71,0.21,0.01}
    \definecolor{citecolor}{rgb}{.12,.54,.11}

    % ANSI colors
    \definecolor{ansi-black}{HTML}{3E424D}
    \definecolor{ansi-black-intense}{HTML}{282C36}
    \definecolor{ansi-red}{HTML}{E75C58}
    \definecolor{ansi-red-intense}{HTML}{B22B31}
    \definecolor{ansi-green}{HTML}{00A250}
    \definecolor{ansi-green-intense}{HTML}{007427}
    \definecolor{ansi-yellow}{HTML}{DDB62B}
    \definecolor{ansi-yellow-intense}{HTML}{B27D12}
    \definecolor{ansi-blue}{HTML}{208FFB}
    \definecolor{ansi-blue-intense}{HTML}{0065CA}
    \definecolor{ansi-magenta}{HTML}{D160C4}
    \definecolor{ansi-magenta-intense}{HTML}{A03196}
    \definecolor{ansi-cyan}{HTML}{60C6C8}
    \definecolor{ansi-cyan-intense}{HTML}{258F8F}
    \definecolor{ansi-white}{HTML}{C5C1B4}
    \definecolor{ansi-white-intense}{HTML}{A1A6B2}

    % commands and environments needed by pandoc snippets
    % extracted from the output of `pandoc -s`
    \providecommand{\tightlist}{%
      \setlength{\itemsep}{0pt}\setlength{\parskip}{0pt}}
    \DefineVerbatimEnvironment{Highlighting}{Verbatim}{commandchars=\\\{\}}
    % Add ',fontsize=\small' for more characters per line
    \newenvironment{Shaded}{}{}
    \newcommand{\KeywordTok}[1]{\textcolor[rgb]{0.00,0.44,0.13}{\textbf{{#1}}}}
    \newcommand{\DataTypeTok}[1]{\textcolor[rgb]{0.56,0.13,0.00}{{#1}}}
    \newcommand{\DecValTok}[1]{\textcolor[rgb]{0.25,0.63,0.44}{{#1}}}
    \newcommand{\BaseNTok}[1]{\textcolor[rgb]{0.25,0.63,0.44}{{#1}}}
    \newcommand{\FloatTok}[1]{\textcolor[rgb]{0.25,0.63,0.44}{{#1}}}
    \newcommand{\CharTok}[1]{\textcolor[rgb]{0.25,0.44,0.63}{{#1}}}
    \newcommand{\StringTok}[1]{\textcolor[rgb]{0.25,0.44,0.63}{{#1}}}
    \newcommand{\CommentTok}[1]{\textcolor[rgb]{0.38,0.63,0.69}{\textit{{#1}}}}
    \newcommand{\OtherTok}[1]{\textcolor[rgb]{0.00,0.44,0.13}{{#1}}}
    \newcommand{\AlertTok}[1]{\textcolor[rgb]{1.00,0.00,0.00}{\textbf{{#1}}}}
    \newcommand{\FunctionTok}[1]{\textcolor[rgb]{0.02,0.16,0.49}{{#1}}}
    \newcommand{\RegionMarkerTok}[1]{{#1}}
    \newcommand{\ErrorTok}[1]{\textcolor[rgb]{1.00,0.00,0.00}{\textbf{{#1}}}}
    \newcommand{\NormalTok}[1]{{#1}}
    
    % Additional commands for more recent versions of Pandoc
    \newcommand{\ConstantTok}[1]{\textcolor[rgb]{0.53,0.00,0.00}{{#1}}}
    \newcommand{\SpecialCharTok}[1]{\textcolor[rgb]{0.25,0.44,0.63}{{#1}}}
    \newcommand{\VerbatimStringTok}[1]{\textcolor[rgb]{0.25,0.44,0.63}{{#1}}}
    \newcommand{\SpecialStringTok}[1]{\textcolor[rgb]{0.73,0.40,0.53}{{#1}}}
    \newcommand{\ImportTok}[1]{{#1}}
    \newcommand{\DocumentationTok}[1]{\textcolor[rgb]{0.73,0.13,0.13}{\textit{{#1}}}}
    \newcommand{\AnnotationTok}[1]{\textcolor[rgb]{0.38,0.63,0.69}{\textbf{\textit{{#1}}}}}
    \newcommand{\CommentVarTok}[1]{\textcolor[rgb]{0.38,0.63,0.69}{\textbf{\textit{{#1}}}}}
    \newcommand{\VariableTok}[1]{\textcolor[rgb]{0.10,0.09,0.49}{{#1}}}
    \newcommand{\ControlFlowTok}[1]{\textcolor[rgb]{0.00,0.44,0.13}{\textbf{{#1}}}}
    \newcommand{\OperatorTok}[1]{\textcolor[rgb]{0.40,0.40,0.40}{{#1}}}
    \newcommand{\BuiltInTok}[1]{{#1}}
    \newcommand{\ExtensionTok}[1]{{#1}}
    \newcommand{\PreprocessorTok}[1]{\textcolor[rgb]{0.74,0.48,0.00}{{#1}}}
    \newcommand{\AttributeTok}[1]{\textcolor[rgb]{0.49,0.56,0.16}{{#1}}}
    \newcommand{\InformationTok}[1]{\textcolor[rgb]{0.38,0.63,0.69}{\textbf{\textit{{#1}}}}}
    \newcommand{\WarningTok}[1]{\textcolor[rgb]{0.38,0.63,0.69}{\textbf{\textit{{#1}}}}}
    
    
    % Define a nice break command that doesn't care if a line doesn't already
    % exist.
    \def\br{\hspace*{\fill} \\* }
    % Math Jax compatability definitions
    \def\gt{>}
    \def\lt{<}
    % Document parameters
    \title{binomial-gamma-hurdle}
    
    
    

    % Pygments definitions
    
\makeatletter
\def\PY@reset{\let\PY@it=\relax \let\PY@bf=\relax%
    \let\PY@ul=\relax \let\PY@tc=\relax%
    \let\PY@bc=\relax \let\PY@ff=\relax}
\def\PY@tok#1{\csname PY@tok@#1\endcsname}
\def\PY@toks#1+{\ifx\relax#1\empty\else%
    \PY@tok{#1}\expandafter\PY@toks\fi}
\def\PY@do#1{\PY@bc{\PY@tc{\PY@ul{%
    \PY@it{\PY@bf{\PY@ff{#1}}}}}}}
\def\PY#1#2{\PY@reset\PY@toks#1+\relax+\PY@do{#2}}

\expandafter\def\csname PY@tok@w\endcsname{\def\PY@tc##1{\textcolor[rgb]{0.73,0.73,0.73}{##1}}}
\expandafter\def\csname PY@tok@c\endcsname{\let\PY@it=\textit\def\PY@tc##1{\textcolor[rgb]{0.25,0.50,0.50}{##1}}}
\expandafter\def\csname PY@tok@cp\endcsname{\def\PY@tc##1{\textcolor[rgb]{0.74,0.48,0.00}{##1}}}
\expandafter\def\csname PY@tok@k\endcsname{\let\PY@bf=\textbf\def\PY@tc##1{\textcolor[rgb]{0.00,0.50,0.00}{##1}}}
\expandafter\def\csname PY@tok@kp\endcsname{\def\PY@tc##1{\textcolor[rgb]{0.00,0.50,0.00}{##1}}}
\expandafter\def\csname PY@tok@kt\endcsname{\def\PY@tc##1{\textcolor[rgb]{0.69,0.00,0.25}{##1}}}
\expandafter\def\csname PY@tok@o\endcsname{\def\PY@tc##1{\textcolor[rgb]{0.40,0.40,0.40}{##1}}}
\expandafter\def\csname PY@tok@ow\endcsname{\let\PY@bf=\textbf\def\PY@tc##1{\textcolor[rgb]{0.67,0.13,1.00}{##1}}}
\expandafter\def\csname PY@tok@nb\endcsname{\def\PY@tc##1{\textcolor[rgb]{0.00,0.50,0.00}{##1}}}
\expandafter\def\csname PY@tok@nf\endcsname{\def\PY@tc##1{\textcolor[rgb]{0.00,0.00,1.00}{##1}}}
\expandafter\def\csname PY@tok@nc\endcsname{\let\PY@bf=\textbf\def\PY@tc##1{\textcolor[rgb]{0.00,0.00,1.00}{##1}}}
\expandafter\def\csname PY@tok@nn\endcsname{\let\PY@bf=\textbf\def\PY@tc##1{\textcolor[rgb]{0.00,0.00,1.00}{##1}}}
\expandafter\def\csname PY@tok@ne\endcsname{\let\PY@bf=\textbf\def\PY@tc##1{\textcolor[rgb]{0.82,0.25,0.23}{##1}}}
\expandafter\def\csname PY@tok@nv\endcsname{\def\PY@tc##1{\textcolor[rgb]{0.10,0.09,0.49}{##1}}}
\expandafter\def\csname PY@tok@no\endcsname{\def\PY@tc##1{\textcolor[rgb]{0.53,0.00,0.00}{##1}}}
\expandafter\def\csname PY@tok@nl\endcsname{\def\PY@tc##1{\textcolor[rgb]{0.63,0.63,0.00}{##1}}}
\expandafter\def\csname PY@tok@ni\endcsname{\let\PY@bf=\textbf\def\PY@tc##1{\textcolor[rgb]{0.60,0.60,0.60}{##1}}}
\expandafter\def\csname PY@tok@na\endcsname{\def\PY@tc##1{\textcolor[rgb]{0.49,0.56,0.16}{##1}}}
\expandafter\def\csname PY@tok@nt\endcsname{\let\PY@bf=\textbf\def\PY@tc##1{\textcolor[rgb]{0.00,0.50,0.00}{##1}}}
\expandafter\def\csname PY@tok@nd\endcsname{\def\PY@tc##1{\textcolor[rgb]{0.67,0.13,1.00}{##1}}}
\expandafter\def\csname PY@tok@s\endcsname{\def\PY@tc##1{\textcolor[rgb]{0.73,0.13,0.13}{##1}}}
\expandafter\def\csname PY@tok@sd\endcsname{\let\PY@it=\textit\def\PY@tc##1{\textcolor[rgb]{0.73,0.13,0.13}{##1}}}
\expandafter\def\csname PY@tok@si\endcsname{\let\PY@bf=\textbf\def\PY@tc##1{\textcolor[rgb]{0.73,0.40,0.53}{##1}}}
\expandafter\def\csname PY@tok@se\endcsname{\let\PY@bf=\textbf\def\PY@tc##1{\textcolor[rgb]{0.73,0.40,0.13}{##1}}}
\expandafter\def\csname PY@tok@sr\endcsname{\def\PY@tc##1{\textcolor[rgb]{0.73,0.40,0.53}{##1}}}
\expandafter\def\csname PY@tok@ss\endcsname{\def\PY@tc##1{\textcolor[rgb]{0.10,0.09,0.49}{##1}}}
\expandafter\def\csname PY@tok@sx\endcsname{\def\PY@tc##1{\textcolor[rgb]{0.00,0.50,0.00}{##1}}}
\expandafter\def\csname PY@tok@m\endcsname{\def\PY@tc##1{\textcolor[rgb]{0.40,0.40,0.40}{##1}}}
\expandafter\def\csname PY@tok@gh\endcsname{\let\PY@bf=\textbf\def\PY@tc##1{\textcolor[rgb]{0.00,0.00,0.50}{##1}}}
\expandafter\def\csname PY@tok@gu\endcsname{\let\PY@bf=\textbf\def\PY@tc##1{\textcolor[rgb]{0.50,0.00,0.50}{##1}}}
\expandafter\def\csname PY@tok@gd\endcsname{\def\PY@tc##1{\textcolor[rgb]{0.63,0.00,0.00}{##1}}}
\expandafter\def\csname PY@tok@gi\endcsname{\def\PY@tc##1{\textcolor[rgb]{0.00,0.63,0.00}{##1}}}
\expandafter\def\csname PY@tok@gr\endcsname{\def\PY@tc##1{\textcolor[rgb]{1.00,0.00,0.00}{##1}}}
\expandafter\def\csname PY@tok@ge\endcsname{\let\PY@it=\textit}
\expandafter\def\csname PY@tok@gs\endcsname{\let\PY@bf=\textbf}
\expandafter\def\csname PY@tok@gp\endcsname{\let\PY@bf=\textbf\def\PY@tc##1{\textcolor[rgb]{0.00,0.00,0.50}{##1}}}
\expandafter\def\csname PY@tok@go\endcsname{\def\PY@tc##1{\textcolor[rgb]{0.53,0.53,0.53}{##1}}}
\expandafter\def\csname PY@tok@gt\endcsname{\def\PY@tc##1{\textcolor[rgb]{0.00,0.27,0.87}{##1}}}
\expandafter\def\csname PY@tok@err\endcsname{\def\PY@bc##1{\setlength{\fboxsep}{0pt}\fcolorbox[rgb]{1.00,0.00,0.00}{1,1,1}{\strut ##1}}}
\expandafter\def\csname PY@tok@kc\endcsname{\let\PY@bf=\textbf\def\PY@tc##1{\textcolor[rgb]{0.00,0.50,0.00}{##1}}}
\expandafter\def\csname PY@tok@kd\endcsname{\let\PY@bf=\textbf\def\PY@tc##1{\textcolor[rgb]{0.00,0.50,0.00}{##1}}}
\expandafter\def\csname PY@tok@kn\endcsname{\let\PY@bf=\textbf\def\PY@tc##1{\textcolor[rgb]{0.00,0.50,0.00}{##1}}}
\expandafter\def\csname PY@tok@kr\endcsname{\let\PY@bf=\textbf\def\PY@tc##1{\textcolor[rgb]{0.00,0.50,0.00}{##1}}}
\expandafter\def\csname PY@tok@bp\endcsname{\def\PY@tc##1{\textcolor[rgb]{0.00,0.50,0.00}{##1}}}
\expandafter\def\csname PY@tok@fm\endcsname{\def\PY@tc##1{\textcolor[rgb]{0.00,0.00,1.00}{##1}}}
\expandafter\def\csname PY@tok@vc\endcsname{\def\PY@tc##1{\textcolor[rgb]{0.10,0.09,0.49}{##1}}}
\expandafter\def\csname PY@tok@vg\endcsname{\def\PY@tc##1{\textcolor[rgb]{0.10,0.09,0.49}{##1}}}
\expandafter\def\csname PY@tok@vi\endcsname{\def\PY@tc##1{\textcolor[rgb]{0.10,0.09,0.49}{##1}}}
\expandafter\def\csname PY@tok@vm\endcsname{\def\PY@tc##1{\textcolor[rgb]{0.10,0.09,0.49}{##1}}}
\expandafter\def\csname PY@tok@sa\endcsname{\def\PY@tc##1{\textcolor[rgb]{0.73,0.13,0.13}{##1}}}
\expandafter\def\csname PY@tok@sb\endcsname{\def\PY@tc##1{\textcolor[rgb]{0.73,0.13,0.13}{##1}}}
\expandafter\def\csname PY@tok@sc\endcsname{\def\PY@tc##1{\textcolor[rgb]{0.73,0.13,0.13}{##1}}}
\expandafter\def\csname PY@tok@dl\endcsname{\def\PY@tc##1{\textcolor[rgb]{0.73,0.13,0.13}{##1}}}
\expandafter\def\csname PY@tok@s2\endcsname{\def\PY@tc##1{\textcolor[rgb]{0.73,0.13,0.13}{##1}}}
\expandafter\def\csname PY@tok@sh\endcsname{\def\PY@tc##1{\textcolor[rgb]{0.73,0.13,0.13}{##1}}}
\expandafter\def\csname PY@tok@s1\endcsname{\def\PY@tc##1{\textcolor[rgb]{0.73,0.13,0.13}{##1}}}
\expandafter\def\csname PY@tok@mb\endcsname{\def\PY@tc##1{\textcolor[rgb]{0.40,0.40,0.40}{##1}}}
\expandafter\def\csname PY@tok@mf\endcsname{\def\PY@tc##1{\textcolor[rgb]{0.40,0.40,0.40}{##1}}}
\expandafter\def\csname PY@tok@mh\endcsname{\def\PY@tc##1{\textcolor[rgb]{0.40,0.40,0.40}{##1}}}
\expandafter\def\csname PY@tok@mi\endcsname{\def\PY@tc##1{\textcolor[rgb]{0.40,0.40,0.40}{##1}}}
\expandafter\def\csname PY@tok@il\endcsname{\def\PY@tc##1{\textcolor[rgb]{0.40,0.40,0.40}{##1}}}
\expandafter\def\csname PY@tok@mo\endcsname{\def\PY@tc##1{\textcolor[rgb]{0.40,0.40,0.40}{##1}}}
\expandafter\def\csname PY@tok@ch\endcsname{\let\PY@it=\textit\def\PY@tc##1{\textcolor[rgb]{0.25,0.50,0.50}{##1}}}
\expandafter\def\csname PY@tok@cm\endcsname{\let\PY@it=\textit\def\PY@tc##1{\textcolor[rgb]{0.25,0.50,0.50}{##1}}}
\expandafter\def\csname PY@tok@cpf\endcsname{\let\PY@it=\textit\def\PY@tc##1{\textcolor[rgb]{0.25,0.50,0.50}{##1}}}
\expandafter\def\csname PY@tok@c1\endcsname{\let\PY@it=\textit\def\PY@tc##1{\textcolor[rgb]{0.25,0.50,0.50}{##1}}}
\expandafter\def\csname PY@tok@cs\endcsname{\let\PY@it=\textit\def\PY@tc##1{\textcolor[rgb]{0.25,0.50,0.50}{##1}}}

\def\PYZbs{\char`\\}
\def\PYZus{\char`\_}
\def\PYZob{\char`\{}
\def\PYZcb{\char`\}}
\def\PYZca{\char`\^}
\def\PYZam{\char`\&}
\def\PYZlt{\char`\<}
\def\PYZgt{\char`\>}
\def\PYZsh{\char`\#}
\def\PYZpc{\char`\%}
\def\PYZdl{\char`\$}
\def\PYZhy{\char`\-}
\def\PYZsq{\char`\'}
\def\PYZdq{\char`\"}
\def\PYZti{\char`\~}
% for compatibility with earlier versions
\def\PYZat{@}
\def\PYZlb{[}
\def\PYZrb{]}
\makeatother


    % Exact colors from NB
    \definecolor{incolor}{rgb}{0.0, 0.0, 0.5}
    \definecolor{outcolor}{rgb}{0.545, 0.0, 0.0}



    
    % Prevent overflowing lines due to hard-to-break entities
    \sloppy 
    % Setup hyperref package
    \hypersetup{
      breaklinks=true,  % so long urls are correctly broken across lines
      colorlinks=true,
      urlcolor=urlcolor,
      linkcolor=linkcolor,
      citecolor=citecolor,
      }
    % Slightly bigger margins than the latex defaults
    
    \geometry{verbose,tmargin=1in,bmargin=1in,lmargin=1in,rmargin=1in}
    
    

    \begin{document}
    
    
    \maketitle
    
    

    
    \subsection{Binomial-Gamma Hurdle Models accounting for zero-inflation
and imperfect
detection}\label{binomial-gamma-hurdle-models-accounting-for-zero-inflation-and-imperfect-detection}

    \subsubsection{Model description}\label{model-description}

We consider here two common problems in ecological data: zero-inflation
and imperfect detection. Approach is presented using simulated data,
that resembles real data such as e.g. species sightings per minute of
effort.

\subparagraph{Dealing with
zero-inflation:}\label{dealing-with-zero-inflation}

Dependent variable, y is semi-continuous (i.e. a point mass in a single
value and a continuous distribution elsewhere). The data generating
process for this type of data can be modelled using a gamma
distribution. The main problem is however that response variable has a
high proportion of zeros (\textgreater{}90\%), which is more than
expected from a gamma distribution, hence it cannot be readily applied.

There are two common methods for dealing with zero-inflated data:

\begin{enumerate}
\def\labelenumi{(\arabic{enumi})}
\tightlist
\item
  Modelling a zero-inflation parameter that represents the probability a
  given 0 comes from the main distribution (say the negative binomial
  distribution) or is an excess 0;
\item
  Modelling the zero and non-zero data with one model and then modelling
  the non-zero data with another. This is often called a hurdle model.
\end{enumerate}

In (1), the response variable is modelled as a mixture of a Bernoulli
distribution (a point mass at zero) and a Poisson distribution (or any
other count distribution supported on non-negative integers). In (2),
the basic idea is that a Bernoulli probability governs the binary
outcome of whether a variable has a zero or positive realization. If the
realization is positive, the hurdle is crossed, and the conditional
distribution of the positives is governed by a truncated-at-zero model.
Hurdle models model the zeros and non-zeros as two separate processes
and can be useful in that they allow you to model the zeros and
non-zeros with different predictors or different roles of the same
predictors.

Zero-inflation models may be more elegant and informative if the same
predictors are thought to contribute to the extra and real zeros. Hurdle
models can be useful in that they allow you to model the zeros and
non-zeros with different predictors or different roles of the same
predictors. Maybe one process leads to the zero/non-zero data and
another leads to the non-zero magnitude.

Here we shall focus on (2) and model the zeros separately from the
non-zeros in a binomial-Gamma hurdle model.

\paragraph{Dealing with imprefect
detection}\label{dealing-with-imprefect-detection}

To isolate the effect of variables that might influence detectability
and to partition out the annual signals, a mixed-effect modelling
framework will be separately applied to each part of the hurdle model. A
varying-intercept model will be fit with the explanatory variables that
are expected to influence detectability (say x1). For the Bernoulli part
of the hurdle model the response is a binary variable
(presence/absence). For the Gamma part the response is the number of
observations. The factor variable `year' is included as a random effect.
This approach is particularly useful when variation among years is of
interest. Random effects are conditional modes calculated as the
difference between the average predicted response for a given set of
fixed-effect values (in this case x1 variable that may influence
detectability) and the response (presence/absence or abundance)
predicted for particular year. These conditional modes were then
extracted for each part of the hurdle and were then included as the
response variable in a series of general linear models that modelled the
effect of x2 variable on y.

    \subsubsection{Load libraries}\label{load-libraries}

    \begin{Verbatim}[commandchars=\\\{\}]
{\color{incolor}In [{\color{incolor}48}]:} \PY{k+kn}{library}\PY{p}{(}lme4\PY{p}{)}
         \PY{k+kn}{library}\PY{p}{(}effects\PY{p}{)}
         \PY{k+kn}{library}\PY{p}{(}optimx\PY{p}{)}
         \PY{c+c1}{\PYZsh{} require(devtools)}
         \PY{c+c1}{\PYZsh{} install\PYZus{}version(\PYZdq{}effects\PYZdq{}, version = \PYZdq{}4.0\PYZhy{}0\PYZdq{})}
\end{Verbatim}


    \begin{Verbatim}[commandchars=\\\{\}]
{\color{incolor}In [{\color{incolor}2}]:} \PY{k+kp}{set.seed}\PY{p}{(}\PY{l+m}{4322}\PY{p}{)}
\end{Verbatim}


    \subsubsection{Read in data}\label{read-in-data}

    \begin{Verbatim}[commandchars=\\\{\}]
{\color{incolor}In [{\color{incolor}49}]:} dat \PY{o}{\PYZlt{}\PYZhy{}} read.csv\PY{p}{(}file \PY{o}{=} \PY{l+s}{\PYZsq{}}\PY{l+s}{data.csv\PYZsq{}}\PY{p}{,} row.names\PY{o}{=}\PY{l+m}{1}\PY{p}{)}
\end{Verbatim}


    Variable y is a response variable, variables x1 and x2 are explanatory
variables. Variable x1 is expected to influence detectibility of y,
variable x2 is expected to relate to y.

    \begin{Verbatim}[commandchars=\\\{\}]
{\color{incolor}In [{\color{incolor}50}]:} \PY{k+kp}{head}\PY{p}{(}dat\PY{p}{)}
\end{Verbatim}


    \begin{tabular}{r|llll}
 y & x1 & x2 & year\\
\hline
	 0.0000000 &  30       & 10.40875  & 1971     \\
	 0.1184611 &  30       & 10.40875  & 1971     \\
	 0.0000000 &  90       & 10.40875  & 1971     \\
	 0.0000000 & 210       & 10.40875  & 1971     \\
	 0.0000000 & 300       & 10.40875  & 1971     \\
	 0.0000000 & 150       & 10.40875  & 1971     \\
\end{tabular}


    
    \subsubsection{Scale data}\label{scale-data}

    \begin{Verbatim}[commandchars=\\\{\}]
{\color{incolor}In [{\color{incolor}51}]:} cols \PY{o}{=} \PY{k+kt}{c}\PY{p}{(}\PY{l+s}{\PYZdq{}}\PY{l+s}{x1\PYZdq{}}\PY{p}{,} \PY{l+s}{\PYZdq{}}\PY{l+s}{x2\PYZdq{}}\PY{p}{)}
         dat\PY{p}{[}\PY{p}{,} \PY{k+kp}{paste0}\PY{p}{(}cols\PY{p}{,} \PY{l+s}{\PYZdq{}}\PY{l+s}{\PYZus{}\PYZdq{}}\PY{p}{,} \PY{l+s}{\PYZdq{}}\PY{l+s}{sc\PYZdq{}}\PY{p}{)}\PY{p}{]} \PY{o}{\PYZlt{}\PYZhy{}} \PY{k+kp}{scale}\PY{p}{(}dat\PY{p}{[} \PY{p}{,}cols\PY{p}{]}\PY{p}{)}
         \PY{k+kp}{summary}\PY{p}{(}dat\PY{p}{)}
\end{Verbatim}


    
    \begin{verbatim}
       y                  x1            x2              year     
 Min.   :0.000000   Min.   :  5   Min.   : 9.207   Min.   :1971  
 1st Qu.:0.000000   1st Qu.: 30   1st Qu.:13.169   1st Qu.:1984  
 Median :0.000000   Median : 60   Median :15.261   Median :1998  
 Mean   :0.002485   Mean   :128   Mean   :14.621   Mean   :1996  
 3rd Qu.:0.000000   3rd Qu.:180   3rd Qu.:16.288   3rd Qu.:2009  
 Max.   :0.133199   Max.   :960   Max.   :18.168   Max.   :2017  
                    NA's   :363   NA's   :61                     
     x1_sc             x2_sc        
 Min.   :-0.8945   Min.   :-2.5651  
 1st Qu.:-0.7126   1st Qu.:-0.6882  
 Median :-0.4944   Median : 0.3032  
 Mean   : 0.0000   Mean   : 0.0000  
 3rd Qu.: 0.3786   3rd Qu.: 0.7894  
 Max.   : 6.0531   Max.   : 1.6802  
 NA's   :363       NA's   :61       
    \end{verbatim}

    
    \subsubsection{Binomial model}\label{binomial-model}

    We fit full model.

    \begin{Verbatim}[commandchars=\\\{\}]
{\color{incolor}In [{\color{incolor}6}]:} \PY{k+kp}{summary}\PY{p}{(}glm\PY{p}{(}\PY{k+kp}{ifelse}\PY{p}{(}dat\PY{o}{\PYZdl{}}y\PY{o}{\PYZgt{}}\PY{l+m}{0}\PY{p}{,}\PY{l+m}{1}\PY{p}{,}\PY{l+m}{0}\PY{p}{)} \PY{o}{\PYZti{}}
                    x1\PYZus{}sc \PY{o}{+}
                    x2\PYZus{}sc \PY{o}{+}
                    year\PY{p}{,}
                    data \PY{o}{=} dat\PY{p}{,}
                family \PY{o}{=} binomial\PY{p}{(}link \PY{o}{=} logit\PY{p}{)}\PY{p}{)}\PY{p}{)}
\end{Verbatim}


    
    \begin{verbatim}

Call:
glm(formula = ifelse(dat$y > 0, 1, 0) ~ x1_sc + x2_sc + year, 
    family = binomial(link = logit), data = dat)

Deviance Residuals: 
    Min       1Q   Median       3Q      Max  
-0.3114  -0.2885  -0.2789  -0.2601   2.8235  

Coefficients:
             Estimate Std. Error z value Pr(>|z|)  
(Intercept) -0.152262  12.235934  -0.012   0.9901  
x1_sc       -0.174758   0.102027  -1.713   0.0867 .
x2_sc       -0.052618   0.080658  -0.652   0.5142  
year        -0.001567   0.006135  -0.255   0.7984  
---
Signif. codes:  0 ‘***’ 0.001 ‘**’ 0.01 ‘*’ 0.05 ‘.’ 0.1 ‘ ’ 1

(Dispersion parameter for binomial family taken to be 1)

    Null deviance: 1455.7  on 4609  degrees of freedom
Residual deviance: 1450.7  on 4606  degrees of freedom
  (365 observations deleted due to missingness)
AIC: 1458.7

Number of Fisher Scoring iterations: 6

    \end{verbatim}

    
    We see that observer related variable (x1) is significant. We try to
isolate its effect for each year. To do so we apply a mixed-effect
modeling framework and fit a varying intercept model. This approach is
useful when we are interested explicitly in variation among and by
groups. Group level variables are specified using a special syntax:
(1\textbar{}year) to fit a linear model with a varying-intercept group
effect using the variable year.

    We include 'year' as random effect with noise variables.

    \begin{Verbatim}[commandchars=\\\{\}]
{\color{incolor}In [{\color{incolor}52}]:} m.bin.full.re \PY{o}{\PYZlt{}\PYZhy{}} glmer\PY{p}{(}\PY{k+kp}{ifelse}\PY{p}{(}dat\PY{o}{\PYZdl{}}y\PY{o}{\PYZgt{}}\PY{l+m}{0}\PY{p}{,}\PY{l+m}{1}\PY{p}{,}\PY{l+m}{0}\PY{p}{)} \PY{o}{\PYZti{}}
                      x1\PYZus{}sc \PY{o}{+}
                      \PY{p}{(}\PY{l+m}{1}\PY{o}{|}year\PY{p}{)} \PY{p}{,}
                    data \PY{o}{=} dat\PY{p}{,}
         \PY{c+c1}{\PYZsh{}            control = glmerControl(optimizer =\PYZsq{}optimx\PYZsq{}, optCtrl=list(method=\PYZsq{}nlminb\PYZsq{})),}
                    family \PY{o}{=} binomial\PY{p}{(}link \PY{o}{=} logit\PY{p}{)}\PY{p}{)}
\end{Verbatim}


    \begin{Verbatim}[commandchars=\\\{\}]
{\color{incolor}In [{\color{incolor}53}]:} \PY{k+kp}{summary}\PY{p}{(}m.bin.full.re\PY{p}{)}
\end{Verbatim}


    
    \begin{verbatim}
Generalized linear mixed model fit by maximum likelihood (Laplace
  Approximation) [glmerMod]
 Family: binomial  ( logit )
Formula: ifelse(dat$y > 0, 1, 0) ~ x1_sc + (1 | year)
   Data: dat

     AIC      BIC   logLik deviance df.resid 
  1457.4   1476.7   -725.7   1451.4     4609 

Scaled residuals: 
    Min      1Q  Median      3Q     Max 
-0.2114 -0.2066 -0.2027 -0.1871  7.2335 

Random effects:
 Groups Name        Variance Std.Dev.
 year   (Intercept) 0.002407 0.04907 
Number of obs: 4612, groups:  year, 44

Fixed effects:
            Estimate Std. Error z value Pr(>|z|)    
(Intercept) -3.27872    0.08461 -38.751   <2e-16 ***
x1_sc       -0.18641    0.09471  -1.968    0.049 *  
---
Signif. codes:  0 ‘***’ 0.001 ‘**’ 0.01 ‘*’ 0.05 ‘.’ 0.1 ‘ ’ 1

Correlation of Fixed Effects:
      (Intr)
x1_sc 0.147 
    \end{verbatim}

    
    Random effects are conditional modes - the difference between the
average predicted response for a given set of fixed-effect values
(observer related variables) and the response predicted for particular
year.

    \begin{Verbatim}[commandchars=\\\{\}]
{\color{incolor}In [{\color{incolor}54}]:} ranef.bin.dat\PY{o}{\PYZlt{}\PYZhy{}}\PY{k+kp}{as.data.frame}\PY{p}{(}ranef\PY{p}{(}m.bin.full.re\PY{p}{)}\PY{p}{)}\PY{p}{[}\PY{k+kt}{c}\PY{p}{(}\PY{l+m}{3}\PY{p}{,}\PY{l+m}{4}\PY{p}{)}\PY{p}{]}
         \PY{k+kp}{colnames}\PY{p}{(}ranef.bin.dat\PY{p}{)}\PY{p}{[}\PY{l+m}{2}\PY{p}{]} \PY{o}{\PYZlt{}\PYZhy{}} \PY{l+s}{\PYZdq{}}\PY{l+s}{reyear\PYZdq{}}
\end{Verbatim}


    \begin{Verbatim}[commandchars=\\\{\}]
{\color{incolor}In [{\color{incolor}55}]:} \PY{k+kp}{head}\PY{p}{(}ranef.bin.dat\PY{p}{)}
\end{Verbatim}


    \begin{tabular}{r|ll}
 grp & reyear\\
\hline
	 1971          &  0.0137390698\\
	 1972          &  0.0035901339\\
	 1973          & -0.0042083611\\
	 1974          & -0.0048717533\\
	 1975          &  0.0022349064\\
	 1976          &  0.0009954729\\
\end{tabular}


    
    \begin{Verbatim}[commandchars=\\\{\}]
{\color{incolor}In [{\color{incolor}56}]:} \PY{c+c1}{\PYZsh{} merge with dat by column year}
         dat.bin.re\PY{o}{\PYZlt{}\PYZhy{}}\PY{k+kp}{merge}\PY{p}{(}dat\PY{p}{,}ranef.bin.dat\PY{p}{,} by.x \PY{o}{=} \PY{l+s}{\PYZdq{}}\PY{l+s}{year\PYZdq{}}\PY{p}{,} by.y \PY{o}{=} \PY{l+s}{\PYZdq{}}\PY{l+s}{grp\PYZdq{}}\PY{p}{)}
\end{Verbatim}


    \begin{Verbatim}[commandchars=\\\{\}]
{\color{incolor}In [{\color{incolor}57}]:} \PY{k+kp}{head}\PY{p}{(}dat.bin.re\PY{p}{)}
\end{Verbatim}


    \begin{tabular}{r|lllllll}
 year & y & x1 & x2 & x1\_sc & x2\_sc & reyear\\
\hline
	 1971       & 0.0000000  &  30        & 10.40875   & -0.7126393 & -1.995932  & 0.01373907\\
	 1971       & 0.1184611  &  30        & 10.40875   & -0.7126393 & -1.995932  & 0.01373907\\
	 1971       & 0.0000000  &  90        & 10.40875   & -0.2761402 & -1.995932  & 0.01373907\\
	 1971       & 0.0000000  & 210        & 10.40875   &  0.5968579 & -1.995932  & 0.01373907\\
	 1971       & 0.0000000  & 300        & 10.40875   &  1.2516065 & -1.995932  & 0.01373907\\
	 1971       & 0.0000000  & 150        & 10.40875   &  0.1603588 & -1.995932  & 0.01373907\\
\end{tabular}


    
    Fit variable x2 against ranef.year.

    \begin{Verbatim}[commandchars=\\\{\}]
{\color{incolor}In [{\color{incolor}58}]:} m.bin.full \PY{o}{\PYZlt{}\PYZhy{}} glm\PY{p}{(}reyear \PY{o}{\PYZti{}}
                           x2\PYZus{}sc\PY{p}{,}
                    data \PY{o}{=} dat.bin.re\PY{p}{,}
                    family \PY{o}{=} gaussian\PY{p}{)}
\end{Verbatim}


    \begin{Verbatim}[commandchars=\\\{\}]
{\color{incolor}In [{\color{incolor}59}]:} \PY{k+kp}{summary}\PY{p}{(}m.bin.full\PY{p}{)}
\end{Verbatim}


    
    \begin{verbatim}

Call:
glm(formula = reyear ~ x2_sc, family = gaussian, data = dat.bin.re)

Deviance Residuals: 
      Min         1Q     Median         3Q        Max  
-0.008938  -0.003452  -0.001559   0.003270   0.013473  

Coefficients:
              Estimate Std. Error t value Pr(>|t|)    
(Intercept)  3.589e-04  7.320e-05   4.903 9.73e-07 ***
x2_sc       -1.235e-04  7.321e-05  -1.687   0.0916 .  
---
Signif. codes:  0 ‘***’ 0.001 ‘**’ 0.01 ‘*’ 0.05 ‘.’ 0.1 ‘ ’ 1

(Dispersion parameter for gaussian family taken to be 2.633023e-05)

    Null deviance: 0.12941  on 4913  degrees of freedom
Residual deviance: 0.12933  on 4912  degrees of freedom
  (61 observations deleted due to missingness)
AIC: -37868

Number of Fisher Scoring iterations: 2

    \end{verbatim}

    
    The model predicted random effect for each year, assuming all other
model variables (if there are any) remain constant.

    \begin{Verbatim}[commandchars=\\\{\}]
{\color{incolor}In [{\color{incolor}60}]:} plot\PY{p}{(}allEffects\PY{p}{(}m.bin.full.re\PY{p}{)}\PY{p}{)}
\end{Verbatim}


    \begin{center}
    \adjustimage{max size={0.9\linewidth}{0.9\paperheight}}{output_27_0.png}
    \end{center}
    { \hspace*{\fill} \\}
    
    \subsubsection{Gamma model}\label{gamma-model}

    \begin{Verbatim}[commandchars=\\\{\}]
{\color{incolor}In [{\color{incolor}61}]:} \PY{c+c1}{\PYZsh{} drop rows with nas}
         no.na.dat \PY{o}{\PYZlt{}\PYZhy{}} na.omit\PY{p}{(}dat\PY{p}{)}
\end{Verbatim}


    We fit Gamma model to the positive part of the model with log link using
a similar set of predictors.

    \begin{Verbatim}[commandchars=\\\{\}]
{\color{incolor}In [{\color{incolor}62}]:} \PY{k+kp}{summary}\PY{p}{(}glm\PY{p}{(}y \PY{o}{\PYZti{}}
                     year \PY{o}{+}
                     x1\PYZus{}sc \PY{o}{+}
                     x2\PYZus{}sc\PY{p}{,}
                     data \PY{o}{=} \PY{k+kp}{subset}\PY{p}{(}no.na.dat\PY{p}{,} y\PY{o}{\PYZgt{}}\PY{l+m}{0}\PY{p}{)}\PY{p}{,}
                     family \PY{o}{=} Gamma\PY{p}{(}link \PY{o}{=} \PY{k+kp}{log}\PY{p}{)}\PY{p}{)}\PY{p}{)}
\end{Verbatim}


    
    \begin{verbatim}

Call:
glm(formula = y ~ year + x1_sc + x2_sc, family = Gamma(link = log), 
    data = subset(no.na.dat, y > 0))

Deviance Residuals: 
     Min        1Q    Median        3Q       Max  
-2.29973  -0.58475   0.05655   0.41737   0.89432  

Coefficients:
              Estimate Std. Error t value Pr(>|t|)
(Intercept) -10.032208   6.618597  -1.516    0.131
year          0.003676   0.003318   1.108    0.269
x1_sc        -0.044201   0.060329  -0.733    0.465
x2_sc         0.014744   0.047531   0.310    0.757

(Dispersion parameter for Gamma family taken to be 0.3448597)

    Null deviance: 104.02  on 169  degrees of freedom
Residual deviance: 103.22  on 166  degrees of freedom
AIC: -593.9

Number of Fisher Scoring iterations: 5

    \end{verbatim}

    
    We separate observer related effect for each year.

    \begin{Verbatim}[commandchars=\\\{\}]
{\color{incolor}In [{\color{incolor}64}]:} m.gamma.full.re \PY{o}{\PYZlt{}\PYZhy{}} glmer\PY{p}{(}y \PY{o}{\PYZti{}}
                      x1\PYZus{}sc \PY{o}{+}
                      \PY{p}{(}\PY{l+m}{1}\PY{o}{|}year\PY{p}{)}\PY{p}{,}
                     data \PY{o}{=} \PY{k+kp}{subset}\PY{p}{(}no.na.dat\PY{p}{,} y\PY{o}{\PYZgt{}}\PY{l+m}{0}\PY{p}{)}\PY{p}{,}
                     control \PY{o}{=} glmerControl\PY{p}{(}optimizer \PY{o}{=}\PY{l+s}{\PYZsq{}}\PY{l+s}{optimx\PYZsq{}}\PY{p}{,} optCtrl\PY{o}{=}\PY{k+kt}{list}\PY{p}{(}method\PY{o}{=}\PY{l+s}{\PYZsq{}}\PY{l+s}{nlminb\PYZsq{}}\PY{p}{)}\PY{p}{)}\PY{p}{,}
                    family \PY{o}{=} Gamma\PY{p}{(}link \PY{o}{=} \PY{k+kp}{log}\PY{p}{)}\PY{p}{)}
\end{Verbatim}


    \begin{Verbatim}[commandchars=\\\{\}]
singular fit

    \end{Verbatim}

    \begin{Verbatim}[commandchars=\\\{\}]
{\color{incolor}In [{\color{incolor}65}]:} \PY{k+kp}{summary}\PY{p}{(}m.gamma.full.re\PY{p}{)}
\end{Verbatim}


    
    \begin{verbatim}
Generalized linear mixed model fit by maximum likelihood (Laplace
  Approximation) [glmerMod]
 Family: Gamma  ( log )
Formula: y ~ x1_sc + (1 | year)
   Data: subset(no.na.dat, y > 0)
Control: glmerControl(optimizer = "optimx", optCtrl = list(method = "nlminb"))

     AIC      BIC   logLik deviance df.resid 
  -593.0   -580.4    300.5   -601.0      166 

Scaled residuals: 
    Min      1Q  Median      3Q     Max 
-1.6776 -0.8645  0.1301  0.8736  1.7613 

Random effects:
 Groups   Name        Variance Std.Dev.
 year     (Intercept) 0.0000   0.0000  
 Residual             0.3362   0.5798  
Number of obs: 170, groups:  year, 39

Fixed effects:
            Estimate Std. Error t value Pr(>|z|)    
(Intercept) -2.69798    0.05806 -46.467   <2e-16 ***
x1_sc       -0.04962    0.06810  -0.729    0.466    
---
Signif. codes:  0 ‘***’ 0.001 ‘**’ 0.01 ‘*’ 0.05 ‘.’ 0.1 ‘ ’ 1

Correlation of Fixed Effects:
      (Intr)
x1_sc 0.175 
convergence code: 0
singular fit

    \end{verbatim}

    
    \begin{Verbatim}[commandchars=\\\{\}]
{\color{incolor}In [{\color{incolor}66}]:} plot\PY{p}{(}allEffects\PY{p}{(}m.gamma.full.re\PY{p}{)}\PY{p}{)}
\end{Verbatim}


    \begin{center}
    \adjustimage{max size={0.9\linewidth}{0.9\paperheight}}{output_35_0.png}
    \end{center}
    { \hspace*{\fill} \\}
    
    \begin{Verbatim}[commandchars=\\\{\}]
{\color{incolor}In [{\color{incolor}67}]:} \PY{c+c1}{\PYZsh{} extract random effect}
         ranef.gamma.dat\PY{o}{\PYZlt{}\PYZhy{}}\PY{k+kp}{as.data.frame}\PY{p}{(}ranef\PY{p}{(}m.gamma.full.re\PY{p}{)}\PY{p}{)}\PY{p}{[}\PY{k+kt}{c}\PY{p}{(}\PY{l+m}{3}\PY{p}{,}\PY{l+m}{4}\PY{p}{)}\PY{p}{]}
\end{Verbatim}


    \begin{Verbatim}[commandchars=\\\{\}]
{\color{incolor}In [{\color{incolor}68}]:} \PY{k+kp}{colnames}\PY{p}{(}ranef.gamma.dat\PY{p}{)}\PY{p}{[}\PY{l+m}{2}\PY{p}{]} \PY{o}{\PYZlt{}\PYZhy{}} \PY{l+s}{\PYZdq{}}\PY{l+s}{reyear\PYZdq{}}
\end{Verbatim}


    \begin{Verbatim}[commandchars=\\\{\}]
{\color{incolor}In [{\color{incolor}69}]:} dat.gamma.re \PY{o}{\PYZlt{}\PYZhy{}} \PY{k+kp}{merge}\PY{p}{(}\PY{k+kp}{subset}\PY{p}{(}no.na.dat\PY{p}{,} y\PY{o}{\PYZgt{}}\PY{l+m}{0}\PY{p}{)}\PY{p}{,}ranef.gamma.dat\PY{p}{,} by.x \PY{o}{=} \PY{l+s}{\PYZdq{}}\PY{l+s}{year\PYZdq{}}\PY{p}{,} by.y \PY{o}{=} \PY{l+s}{\PYZdq{}}\PY{l+s}{grp\PYZdq{}}\PY{p}{)}
\end{Verbatim}


    \begin{Verbatim}[commandchars=\\\{\}]
{\color{incolor}In [{\color{incolor}70}]:} \PY{c+c1}{\PYZsh{} fit the rest of the variables against ranef.year}
         m.gamma.full \PY{o}{\PYZlt{}\PYZhy{}} glm\PY{p}{(}reyear \PY{o}{\PYZti{}}
                     x2\PYZus{}sc\PY{p}{,}
                    data \PY{o}{=} dat.gamma.re\PY{p}{,}
                    family \PY{o}{=} gaussian\PY{p}{)}
\end{Verbatim}


    \begin{Verbatim}[commandchars=\\\{\}]
{\color{incolor}In [{\color{incolor}71}]:} \PY{k+kp}{summary}\PY{p}{(}m.gamma.full\PY{p}{)}
\end{Verbatim}


    
    \begin{verbatim}

Call:
glm(formula = reyear ~ x2_sc, family = gaussian, data = dat.gamma.re)

Deviance Residuals: 
   Min      1Q  Median      3Q     Max  
     0       0       0       0       0  

Coefficients:
            Estimate Std. Error t value Pr(>|t|)
(Intercept)        0          0      NA       NA
x2_sc              0          0      NA       NA

(Dispersion parameter for gaussian family taken to be 0)

    Null deviance: 0  on 169  degrees of freedom
Residual deviance: 0  on 168  degrees of freedom
AIC: -Inf

Number of Fisher Scoring iterations: 1

    \end{verbatim}

    
    \subsubsection{References}\label{references}

    \begin{itemize}
\tightlist
\item
  https://www.ssc.wisc.edu/sscc/pubs/MM/MM\_DiagInfer.html
\item
  http://rstudio-pubs-static.s3.amazonaws.com/5691\_192685385fc445c9b3fb1619960a20e2.html
\item
  http://pj.freefaculty.org/guides/stat/Regression-GLM/Gamma/GammaGLM-01.pdf
\item
  https://stats.stackexchange.com/questions/81457/what-is-the-difference-between-zero-inflated-and-hurdle-distributions-models
\item
  http://seananderson.ca/2014/05/18/gamma-hurdle.html
\item
  https://ms.mcmaster.ca/\textasciitilde{}bolker/R/misc/modelDiag.html
\item
  https://stats.idre.ucla.edu/r/dae/logit-regression/
\item
  https://stats.idre.ucla.edu/other/mult-pkg/faq/general/faq-how-do-i-interpret-odds-ratios-in-logistic-regression/
\item
  http://environmentalcomputing.net/interpreting-coefficients-in-glms/
\item
  https://www.sciencedirect.com/science/article/pii/S0167947308000169
\item
  https://www.cambridge.org/core/services/aop-cambridge-core/content/view/S0021859611000608
\item
  https://link.springer.com/article/10.1007/s10742-017-0169-9
\item
  https://stats.stackexchange.com/questions/96972/how-to-interpret-parameters-in-glm-with-family-gamma/126225
\item
  https://stats.stackexchange.com/questions/161216/backtransform-coefficients-of-a-gamma-log-glmm
\end{itemize}


    % Add a bibliography block to the postdoc
    
    
    
    \end{document}
